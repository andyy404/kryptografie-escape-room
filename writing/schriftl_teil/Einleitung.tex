% !TEX root = MA.tex
\section{Einleitung}

Kryptografie ist besonders faszinierend, da jeder täglich davon Gebrauch macht und der Begriff trotzdem kaum jemandem etwas sagt. Aber gäbe es die Kryptografie nicht, wäre es unmöglich, einen sicheren elektronischen Zahlungsverkehr aufzubauen, vertrauliche Daten auf einem Computer zu speichern oder das Internet in einem Ausmass zu nutzen, wie es heute für viele Menschen Gewohnheit ist.

Doch die Kryptografie gibt es nicht erst seit dem Internet und den digitalen Medien. Ursprünglich kam sie aus dem Kriegswesen der Antike. Dabei war die Hauptanwendung, Befehle sicher an Verbündete übermitteln zu können und dabei die Gefahr, dass etwaige Feinde die Nachricht abfangen und mitlesen, zu minimieren. Verschlüsselungsverfahren und Geheimschriften wurden auch häufig für andere vertrauliche Informationen benutzt, wie etwa Rezepte für Heilmittel und Liebesbriefe. Kryptografie wurde in vielen Kriegen verwendet und hat auch in den meisten einen massgeblichen Unterschied gemacht, wie etwa im zweiten Weltkrieg, als es Alan Turing, einem britischen Mathematiker und Kryptanalyst gelang, das Enigma-Verschlüsselungssystem der Deutschen zu knacken. Dabei hat er unter anderem auch die Grundbausteine für die moderne Informatik und moderne Computer gelegt. \cite{wikipedia:alan_turing}

Heutzutage findet Kryptografie vor allem in der digitalen Welt Gebrauch, wobei es nicht mehr nur darum geht, Nachrichten in ein unleserliches Format zu verschlüsseln, sondern auch um Datenschutz und Datensicherheit. Dabei gibt es folgende Hauptziele: \cite{wikipedia:kryptographie}

\begin{itemize}
\item{Es sollen nur dazu berechtigte Personen in der Lage sein, Daten zu lesen oder sonstige Informationen über ihren Inhalt zu erlangen.}
\item{ Es soll nachweisbar sein, dass die Daten vollständig und unverändert sind.}
\item{Es soll eindeutig und auch gegenüber Dritten nachweisbar sein, wer der Urheber der Daten ist und diese Urheberschaft soll unbestreitbar sein.}
\end{itemize}

Das Ziel dieser Maturaarbeit ist, einige der simpleren Verschlüsselungsmethoden aufzuarbeiten und in Form eines Spiels verständlich zu machen. Für die Programmierung dieses Spiels wird Python (Version 3.9.5), eine als übersichtlich und einfach zu erlernend geltende Programmiersprache verwendet. Weiter wird die Pygame-Bibliothek (Version 2.1.2) genutzt, eine Erweiterung für Python spezifisch für das Programmieren von Videospielen, die viele der sonst anstrengenderen Teile des Spiele-Programmierens, wie zum Beispiel die Grafik, einfacher macht. Zuletzt wird Visual Studio Code, eine integrierte Entwicklerumbgebung von Microsoft, benutzt, die unter anderem Python unterstützt.
