% !TEX root = MA.tex
\section{Einleitung}

Kryptografie finde ich besonders faszinierend, da jeder täglich davon Gebrauch macht und der Begriff trotzdem kaum jemandem etwas sagt. Aber gäbe es die Kryptografie nicht, wäre es unmöglich, einen sicheren elektronischen Zahlungsverkehr aufzubauen oder vertrauliche Daten auf einem Computer zu speichern. 

Doch die Kryptografie gibt es nicht erst seit dem Internet und den digitalen Medien. Ursprünglich kommt sie aus dem Kriegswesen der Antike. Dabei ging es hauptsächlichlich darum, Befehle sicher an Verbündete vermitteln zu können, ohne dass etwaige Feinde die Nachricht abfangen und mitlesen können. Kryptografie wurde in den meisten, wenn nicht sogar allen Kriegen verwendet und hat indirekt sogar zur Erfindung des modernen Computers beigetragen

In dieser Maturaarbeit versuche ich einige der simpleren Verschlüsselungsmethoden aufzuarbeiten und in der Form eines Spiels verständlich zu machen. Für die Programmierung dieses Spiels verwende ich Python (Version 3.9.5), eine Anfang der 1990-er Jahre entwickelte und als übersichtlich und einfach zu erlernend geltende Porgrammiersprache. Weiter verwende ich die Pygame-Bibliothek (Version 2.0.1), eine Erwiterung für Python spezifisch für das Programmieren von Videospielen, die viele der sonst anstrengenderen Teile des Programmierens, wie zum Beispiel die Grafik, einfacher macht. Zusätzlich benutze ich Visual Studio Code, eine integrierte Entwicklerumbgebung, die unter anderem Python unterstützt.