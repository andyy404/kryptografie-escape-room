% !TEX root = MA.tex
\section{Behandelte kryptografische Methoden}

\subsection{Geheimtinte}

Die einfachste Art, einen geheimen Text zu vermitteln, ist, wenn er gar nicht erst gesehen werden kann. Dies kann mit verschiedenen Chemikalien erreicht werden, die unter Normalbedingungen farblos sind und entweder durch Hitze, UV-Licht oder chemische Behandlungen sichtbar werden. Diese Art von \enquote{Verschlüsselung} gehört nicht zum Themenbereich der Kryptografie, sondern dem der Steganografie. Diese beschäftigt sich mit dem Verstecken geheimer Nachrichten. Dies geschieht zum Beispiel mit Geheimtinten oder indem die Nachricht in einer anderen nicht geheimen Nachricht versteckt wird. Damit ein \enquote{leeres} Blatt Papier nicht verdächtig erschien, wurde oft eine unschuldige Nachricht darauf geschrieben, mit der Geheimnachricht entweder zwischen den Zeilen oder am Rand des Blattes.

Einer der frühesten Autoren, der unsichtbare Tinte erwähnt, ist Aineias Taktikos, ein griechischer Stratege und Militärschriftsteller aus dem 4. Jahrhundert v. Chr. Im alten Griechenland und im römischen Reich wurden Geheimtinten noch zu Kriegszwecken genutzt, da es damals noch keine elektronische Datenübertragung gab und Nachrichten zumeist niedergeschrieben wurden, wofür sich Geheimtinte sehr eignete. Später im 17. bis 19. Jahrhundert erlebten Geheimtinten einen Popularitätsschub, da sie oft benutzt wurden, um geheime Liebesbriefe zu schreiben.

Die frühesten Formen von unsichtbaren Tinten waren Milch, über die man, wenn sie angetrocknet war, Kohlestaub oder Asche blasen konnte, der an der getrocktenten Milch kleben blieb und Urin und Fruchtsäfte, deren enthaltene Kohlenhydrate (Kohlenstoff) bei Erwärmung verkohlen und dunkler werden, wie man es auch zum Beispiel beim Verbrennen von Holz beobachten kann. Später begann man auch in der Chemie mehr Möglichkeiten für Geheimtinten zu finden, die zum Teil auch nur auf bestimmte andere Chemikalien reagierten. Ein Besipiel dafür ist Phenolphthaleinlösung, ein bekannter pH-Wert Indikator, der bei einem pH-Wert von 0 bis 8,2 farblos ist. Beim Bestreichen mit einer stärkeren Base wie zum Beispiel einer Ammoniaklösung, wird das Phenolphthalein pink.

Heutzutage werden Geheimtinten fast ausschliesslich von Kindern für \enquote{geheime} Nachrichten genutzt, da sie in der digitalen Welt kaum Anwendung finden und überhaupt nur sehr schwierig anzuwenden wären, da sie sich fast nur für physisch niedergeschriebene Nachrichten eignen.

\newpage % CAESAR
\subsection {Caesar-Verschlüsselung}

Caesar ist eine der simpelsten Verschlüsselungen und wird heute fast ausschliesslich benutzt, um Kryptographie einfach zu erklären, da es für heutige Verwendungen viel zu unsicher ist. 

Für die Verschlüsselung wird zusätzlich zum Klartextalphabet ein Geheimalphabet benutzt, wobei jeder Buchstabe jeweils einem bestimmten verschlüsselten Buchstaben entspricht.

Das verschlüsselte Alphabet erhält man, indem man die Zeichen des lateinischen Alphabets um eine bestimmte Anzahl verschiebt (wobei der Anfang des Klartextalphabets zyklisch am Ende des Alphabets angefügt wird, sodass es aufgeht) und jeweils den Klartextbuchstaben mit dem resultierenden Geheimtextbuchstaben ersetzt. Um anzugeben um wie viel das ALphabet verschoben wurde, wird entweder die Anzahl der Stellen oder der Schlüsselbuchstabe (der Buchstabe, durch den A ersetzt wurde) angegeben. 

Beispiel für eine Verschiebung um 10 Buchstaben:

Somit wird aus dem Wort «BEISPIEL» «LOSCZSOV», wenn es verschlüsselt wird. Um den Geheimtext wieder zu entschlüsseln wird der ganze Vorgang rückwärts angewandt. Der Schlüssel (in diesem Fall K) wird dabei von A ersetzt.

Die Caesar-Verschlüsselung kann auch mathematisch dargestellt werden, indem man jedem der 26 Buchstaben eine Zahl zuordnet (A = 0, B = 1, …, Z = 25). Mit diesen Zahlen kann man die Caesar-Verschlüsselung als ganz einfache Addition darstellen. Dazu wird zum Wert des Klartextbuchstabens K einfach der Wert des Schlüsselbuchstabens S addiert.

Da es aber Fälle gibt, in denen das Resultat grösser als 25 ist und es keinen Buchstaben mit einem so hohen Wert gibt, muss auf das Resultat eine Modulo-26 Rechnung angewandt werden. Dabei wird der Rest einer Division durch 26 berechnet.

Somit ist die Caesar-Verschlüsselung mathematisch definiert als:

encrypt

Die dazugehörende Entschlüsselung eines Geheimtextbuchstabens G entspricht dann:

decrypt

%Sicherheit
Da die Caesar-Verschlüsselung eine monoalphabetische Verschlüsselung ist, das heisst, jeder Klartextbuchstabe im Klartextalphabet genau einem Geheimbuchstaben im Geheimalphabet entspricht, kann sie durch Statistik sehr leicht geknackt werden.
Jede Sprache hat eine charakteristische Verteilung der Buchstaben, die leicht in einem Graph aufgezeichnet werden können:

(Graph von der Buchstabenverteilung in der deutschen Sprache einfügen)

Wenn also das ganze Alphabet um beispielsweise 10 Stellen verschoben wird, sieht die Verteilung folgendermassen aus:

(Graph der Buchstabenverteilung 10 Stellen verschoben einfügen)

Da der Verlauf des Graphen immer noch derselbe ist und man weiss, dass E der häufigste Buchstabe ist, kann man nun schliessen, dass das O des Geheimalphabets dem E des Klartextalphabets entspricht. So kann man die Verschiebung berechnen und daraus das Klartextalphabet ableiten.

Da die Buchstabenverteilung jedoch erst in genügend langen Texten genau ist, sollten einzelne Wörter und kurze Sätze in dieser Hinsicht noch einigermassen sicher sein.

Allerdings ist eine weitere Schwäche der Caesar-Verschlüsselung jedoch, dass es nur 25 mögliche Schlüssel gibt, man also spätestens nach 25 Versuchen den Klartext erhält. Vor dieser Angehensweise sind dann auch kurze Sätze und einzelne Wörter nicht mehr sicher.

Im Englischen gibt es zusätzlich noch das Problem, dass es nur zwei Möglichkeiten gibt für Wörter mit einem Buchstaben (\enquote{I} = ich und \enquote{a} = ein), was das Knacken noch zusätzlich beschleunigt, besonders da beides eher häufige Wörter sind.

\textbf{Atbasch} ist eine ursprünglich auf dem hebräischen Alphabet basierende Variante der Caesar-Verschlüsselung, die auch als umgekehrte Caesar-Verschlüsselung bezeichnet wird, denn statt dass die Buchstaben um eine bestimmte Anzahl Stellen verschoben werden, ist das Geheimalphabet lediglich das Klartextalphabet aber rückwärts, sodass A zu Z wird, B zu Y, und so weiter.

Der Name Atbasch leitet sich dabei von den ersten zwei Buchstabenpaaren ab, die einander ersetzen (Aleph mit Taw und Beth mit Schin)
Speziell an Atbasch ist, dass zum Entschlüsseln der gleiche Prozess benutzt werden kann wie zum Verschlüsseln, da die Buchstaben symmetrisch ausgetauscht werden.

(Beispiel einfügen)

\textbf{ROT13} ist eine weitere Variante der Caesar-Verschlüsselung die den gleichen Prozess zum Verschlüsseln und Entschlüsseln benutzt. Hier sind die Buchstaben zwar wie in der normalen Caesar-Verschlüsselung verschoben, aber genau um ein halbes Alphabet, also 13 Stellen. Wenn man also ein Buchstabe verschlüsselt (um 13 Stellen verschiebt) und entschlüsselt (um weitere 13 Stellen verschiebt), hat man den Buchstaben um insgesamt 26 Stellen, also ein ganzes Alphabet verschoben, womit man wieder beim Ausgangsbuchstaben landet.

(Beispiel einfügen)

\newpage %VIGENERE			
\subsection{Vigenère-Verschlüsselung}
Anders als bei der Cäsar-Verschlüsselung wird bei der Vigenère-Verschlüsselung ein Schlüssel in Kombination mit 26 Geheimalphabeten benutzt. Dabei wird der Schlüssel so oft wiederholt, bis er die Länge der zu verschlüsselnden Nachricht deckt.

Geschichte
sdftadfgdfg

Sicherheit
sdftsdfgdfg

Varianten
Die \textbf{Trithemius-Verschlüsselung} ist der Vorläufer der Vigenère-Verschlüsselung und wurde vom deutschen Autor und Mönch Johannes Trithemius im frühen 16. Jahrhundert zusammen mit der Tabula Recta erfunden. Für diese Verschlüsselung benutzt man auch die Tabula Recta aber im Vergleich zu der normalen Vigenère-Verschlüsselung benutzt man keinen Schlüssel, sondern man rückt bei jedem Buchstaben eine Zeile der Tabula Recta weiter nach unten. Im Grunde genommen ist die Trithemius-Verschlüsselung also eine Vigenère-Verschlüsselung mit einem fixen Schlüssel ABCDEFGHIJKLMNOPQRSTUVWXYZ.

Da die Vigenère-Verschlüsselung nicht reziprok ist, das heisst dass das Vorgehen des Verschlüsselns nicht das gleiche ist wie das Vorgehen beim Entschlüsseln, kann man auch «in die falsche Richtung» verschlüsseln und den Klartext sozusagen «entschlüsseln», sodass man es nachher mit der normalen Verschlüsselungstechnik wieder entschlüsseln kann. Dies wird als Beaufort Variante bezeichnet und ist nicht zu verwechseln mit dem Beaufort-Chiffre.



