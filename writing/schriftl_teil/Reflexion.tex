% !TEX root = MA.tex
\section{Reflexion und Ausblick}
\subsection{Rückblick}
Grundsätzlich bin ich mit meiner Arbeit zufrieden, allerdings gibt es auch noch viel Luft nach oben.

Mein Hauptproblem war das Zeitmanagement. Es fiel mir besonders am Anfang der Arbeit sehr schwer einzuschätzen, welcher Teil der Arbeit wie viel Zeit in Anspruch nimmt. So hatte ich zum Beispiel im Konzept geplant, die Programme und den schriftlichen Teil bis Ende der Sommerferien fertig geschrieben zu haben.

Auch die feinere Aufteilung der Arbeit bereitete mir beim Programmieren grosse Probleme. Anfangs hatte ich eine grosse Programmdatei mit etwas über tausend Zeilen, in der es schwierig war, die Übersicht zu behalten, besonders dann, wenn ich an mehreren Stellen gleichzeitig arbeiten musste. Das Ganze dann nach allen Regeln der Kunst in einzelne Skripte aufzuteilen, die trotzdem noch korrekt miteinander funktionieren, kostete mich ziemlich viel Zeit. Auch war ich ziemlich überfordert damit, alles bereits Programmierte noch einmal durchgehen und überprüfen zu müssen und gleichzeitig alles in Klassen umzuprogrammieren, die ich von den Funktionen und Syntax her noch nicht lange und auch noch nicht gut kannte. Das objektorientierte Programmieren (das Arbeiten mit Klassen und Objekten) war insgesamt noch ganz neu für mich und ich musste es zuerst einmal lernen.

Auch konnte ich meine ursprünglichen Vorstellungen nicht ganz verwirklichen. So war am Anfang die Idee, sich im Raum auch umdrehen oder sogar umhergehen zu können. Darauf musste ich leider verzichten, weil es programmatisch zu schwierig umzusetzen war.

Zuletzt war es eine zusätzliche Herausforderung, meine Dateien mit GitHub zu verwalten, sodass ich, im Falle eines Fehlers oder dem Verlust einer Datei auf alle hochgeladenen Versionen der Datei zugreifen und sie herunterladen könnte. Davon profitierte ich tatsächlich mehrmals, zum Beispiel als ich beim Aufteilen der Hauptdatei in einzelne Skripte meine Hauptdatei mit einem anderen Skript ersetzte und es so speicherte.

Rückblickend bin ich immer noch überzeugt von meiner Themenwahl, da mich die Kryptografie und das Spiele-Programmieren bis heute interessieren und ich deshalb immer mit Interesse arbeiten konnte.

\subsection{Ausblick}
Es gibt viele Dinge, die ich in meinem Spiel noch ausbauen oder besser machen könnte. Einerseits wäre das zum Beispiel schönere Grafik, etwas mehr als nur die schwarz-auf-weissen oder weiss-auf-schwarzen Linien. Bessere Grafik hätte mich zu viel Zeit gekostet und ich hätte dann wahrscheinlich eine ganze Maturaarbeit allein über die Grafik machen können, was nicht mein Ziel war. Ausserdem könnte man an einigen Orten auch Animationen verwenden.

Was ebenfalls noch fehlt für ein «vollständiges» Game, wären Musik und sonstige Geräusche, die ich nicht mit einbaute, da ich einerseits auch dafür nicht die Zeit gehabt hätte und andererseits fand ich sie für ein Game wie dieses auch nicht so wichtig.

Eine gute Möglichkeit für aufbauende und weiterführende Projekte steht durch die einfache Erweiterbarkeit des Spiels zur Verfügung. So könnte man mit wenig Aufwand zusätzliche Räume hinzufügen oder bereits Bestehende modifizieren.

Zuletzt könnte man beispielsweise auch die originale Idee, dass man sich im Raum frei bewegen kann, umsetzen oder Kontrolle mit der Tastatur ermöglichen, besonders bei den Türen, and denen man Wörter eingeben muss.

