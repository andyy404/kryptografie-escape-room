% !TEX root = MA.tex
\section{Das Escape Room Spiel}

Achtung: Dieses Kapitel enthält Informationen, wie das Spiel bedient und auch durchgespielt werden kann. Wer die Lösungen selbst herausfinden möchte, sollte dieses Kapitel vorerst überspringen.

\subsection{Benutzeroberfläche}
Wenn das Spiel geöffnet wird, erscheint als erstes das Hauptmenü des Spiels.

\begin{figure}[htb]
	\centering
		\includegraphics[height=0.45\textwidth]{\BILDER hauptmenue}
 	\caption{Das Hauptmenu}
  \label{fig:hauptmenu}
\end{figure}

 
Im Hauptmenü gibt es zwei Knöpfe, den Startknopf (1), der das Spiel startet und den Spieler in den ersten Raum setzt und den Beenden-Knopf (2), der das Spiel stoppt und das Programm schliesst.

Das Spiel wird ausschliesslich mit der Maus gespielt, Tastatureingaben sind nicht vorgesehen. \newpage

\begin{figure}[h!]
	\centering
		\includegraphics[height=0.45\textwidth]{\BILDER room1}
 	\caption{Die Hauptansicht des ersten Raumes}
  \label{fig:hauptansicht}
\end{figure}

Wenn das Spiel gestartet wird, erscheinen die Einrichtung des ersten Raumes (5), der Pausenknopf (3) und ein Timer (4). Das Ziel ist es, in allen Räumen das korrekte Passwort oder den korrekten Code einzugeben, bevor der Timer abläuft. Mit den meisten Objekten aus der Hauptansicht (5) kann interagiert werden. So kann zum Beispiel auf das Regal ganz links geklickt werden, was eine neue, detaillierte Ansicht des Objekts anzeigt.

\begin{figure}[h!]
	\centering
		\includegraphics[height=0.45\textwidth]{\BILDER interaction_screen}
 	\caption{Die Detailansicht des Regals mit der UV-Taschenlampe}
  \label{fig:detailansicht}
\end{figure}

In den detaillierten Ansichten von Objekten wird der Pausenknopf durch einen Knopf (6) ersetzt, mit dem man zur Hauptansicht zurückwechseln kann. In den meisten Detailansichten von Objekten besteht die Möglichkeit, weiter damit zu interagieren, so wie man beispielsweise hier die UV-Taschenlampe anklicken kann, wodurch sie aufgehoben wird.

\begin{figure}[h!]
	\centering
		\includegraphics[height=0.45\textwidth]{\BILDER tools}
 	\caption{Die Hauptansicht mit aufgehobenen Werkzeugen}
  \label{fig:werkzeuge}
\end{figure}

Aufgehobene Werkzeuge werden in der rechten unteren Bildschirmecke in Form von Knöpfen angezeigt. In der Abbildung \ref{fig:werkzeuge} trägt der Spieler sowohl die UV-Taschenlampe (11) als auch die Chiffrierscheibe (10). Mit Klicken auf eines der Werkzeuge kann dieses benutzt werden.

\begin{figure}[h!]
	\centering
		\includegraphics[height=0.45\textwidth]{\BILDER chiffrierscheibe}
 	\caption{Die Chiffrierscheibe}
  \label{fig:chiffrierscheibe}
\end{figure}

Im Falle der Chiffrierscheibe öffnet sich eine grosse Ansicht der Chiffrierscheibe, die auch benutzt werden kann. Dazu klickt man mit der Maus auf die innere Scheibe (12) und hält sie mit der Maus fest, während man daran dreht. Auch in dieser Ansicht befindet sich ein Knopf (6) mit der zurück zur Hauptansicht geschalten werden kann.

\begin{figure}[h!]
	\centering
		\includegraphics[height=0.45\textwidth]{\BILDER uvlampe}
 	\caption{Die Benutzung der UV-Taschenlampe}
  \label{fig:uvlampe}
\end{figure}

Die Taschenlampe funktioniert ausschliesslich, solange das Licht im Raum ausgeschaltet ist, was man jeweils durch Anklicken der Lampe (13) im Zimmer macht. Die Taschenlampe muss ebenfalls mit einem Mausklick «gepackt» werden. Solange man sie festhält, kann man den Mauszeiger bewegen und so den Lichtschein der Lampe (14) kontrollieren. 

Mit dem Pausenknopf (3) der Hauptansicht kann das Spiel sowie der Timer pausiert werden, woraufhin das Pausenmenü angezeigt wird.

\begin{figure}[htbp]
	\centering
		\includegraphics[height=0.45\textwidth]{\BILDER pausemnue}
 	\caption{Das Pausenmenü}
  \label{fig:pausemenu}
\end{figure}

Der Timer (4) wird weiterhin angezeigt, ist aber eingefroren. Weiter erscheint der Knopf zum Beenden des Spiels (2) wie im Hauptmenü. Mit dem Fortsetzen-Knopf (9) kann man ins Spiel zurückkehren, um dort weiterzuspielen wo aufgehört wurde. Dadurch läuft auch der Timer weiter. Mit dem Hauptmenü-Knopf (8) kann man ins Hauptmenü zurückkehren, wo ein neues Spiel gestartet werden kann.

\subsection{Das Programm}

\subsubsection{Vorteile und Nachteile von Python}
Python ist eine vergleichsweise simple Programmiersprache, die einfach gelernt werden kann, was einigen Aufwand gespart hat. Allerdings ist Python keine allzu beliebte Programmiersprache in der Spieleentwicklung, da sie bei komplexeren Spielen schnell an die Grenzen ihrer Leistungsfähigkeit stossen kann und es auch fast keine Unterstützung für dreidimensionale Spiele gibt.
 
Die Entscheidung, kein Game-Engine und stattdessen eine Python-Bibliothek zu benutzen, war sehr bewusst, da es im Kern des Informatikteils dieser Maturaarbeit auch darum geht zu lernen, wie Videospiele fundamental funktionieren. 

\subsubsection{Programmatische Umsetzung der kryptografischen Methoden}
Für alle im Kapitel 3 vorgestellten Verschlüsselungsmethoden wurde je eine Klasse mit einer Verschlüsselungs- und einer Entschlüsselungsfunktion programmiert. Dafür wurden zuerst zwei allgemeine Funktionen \verb+letterify()+ und \verb+numberify()+ geschrieben, die Zahlenwerte in ihre entsprechenden Buchstaben, beziehungsweise Buchstaben in ihre entsprechenden Zahlenwerte zwischen 0 und 25 umwandeln können.So wird es ermöglicht mathematisch mit den Buchstaben umzugehen.

\begin{Verbatim}
def numberify(letter):
   integer = ord(letter)

    if 65 <= integer <= 90:
        integer -= 65

    if 97 <= integer <= 122:
        integer -= 97

    return integer
\end{Verbatim}

\newpage

\begin{Verbatim}
def letterify(integer):
    integer += 97
    letter = chr(integer)

    return letter
\end{Verbatim}

Alle der zu den Verschlüsselungstechniken geschriebenen Klassen sind im Grunde gleich aufgebaut. Sie enthalten zwei Funktionen: \verb+encrypt()+, die Verschlüsselung und \verb+decrypt()+, die Entschlüsselung. Dabei werden den Funktionen als Parameter jeweils der Klar- beziehungsweise Geheimtext und falls die Verschlüsselungsmethode einen Schlüssel benutzt auch der Schlüssel übergeben. Danach wird eine Schleife so oft ausgeführt, bis sie alle Buchstaben des Klartextes durchgearbeitet hat. 

\begin{Verbatim}
class Caesar():
    def encrypt(self, plaintext, key):
        ciphertext = ""
        for char in plaintext:
            ciphertext + = letterify((numberify(char) + key) % 26)

        return ciphertext

    def decrypt(self, ciphertext, key):
        plaintext = "" 
        for char in ciphertext:
            plaintext + = letterify((numberify(char) - key) % 26)

        return plaintext
\end{Verbatim}

Im Beispiel dieser Cäsar-Verschlüsselung wird am Anfang die Variable \verb+ciphertext+, ein String in den Schritt für Schritt der Geheimtext hineingeschrieben wird, initialisiert. Danach wird mit einer \verb+for+-Schleife jeder Buchstabe im Klartext einzeln verschlüsselt und an den Geheimtext angefügt. Für den Code der Verschlüsselung wird die mathematische Darstellung der Cäsar-Verschlüsselung benutzt. Zuerst wird der Buchstabe mit \verb+numberify()+ in eine Zahl zwischen 0 und 25 umgewandelt, zu dem dann der Schlüsselwert \verb+key+ addiert wird. Das Resultat der Rechnung wird mit einer modulo-26 Rechnung (in Python dargestellt mit einem Prozentzeichen) wieder in eine Zahl zwischen 0 bis 25 umgewandelt, für den Fall, dass das Resultat durch die Addition grösser als 25 wurde. Die daraus resultierende Zahl wird mit letterify() wieder in einen Buchstaben umgewandelt und hinten an den Geheimtext angefügt.


\newpage

\begin{Verbatim}
class VariantBeaufort():
    def encrypt(self, plaintext, key):
        ciphertext = ""
        for i, char in enumerate(plaintext):
            ciphertext += letterify((numberify(char) 
				- numberify(key[i % len(key)])) % 26)
        
        return ciphertext

    def decrypt(self, ciphertext, key):
        plaintext = ""
        for i, char in enumerate(ciphertext):
            plaintext += letterify((numberify(char) 
				+ numberify(key[i % len(key)])) % 26)
        
        return plaintext
\end{Verbatim}

Eine etwas kompliziertere Verschlüsselung zum Programmieren war beispielsweise die Beaufort-Variante, da bei dieser nicht einfach ein konstanter Schlüssel addiert werden konnte, sondern aus dem Schlüssel erst einmal mit dem Index des Klartextbuchstabens der richtige Schlüsselbuchstabe herausgenommen werden musste. Da es aber Fälle gibt, in denen der Schlüssel kürzer ist als der Klartext, muss auch auf den Index eine Modulo-Rechnung angewendet werden, aber diesmal eine mit dem Wert der Anzahl Stellen, die der Schlüssel hat. So wird der Schlüssel so oft wiederholt, bis er die Länge des Klartextes abdeckt.

\subsubsection{Einfache Erweiterung}
Da die meisten Skripte dieses Projekts objektorientiert und in Form von Klassen programmiert wurden, ist es relativ einfach, selbst neue Räume anzufügen. Sobald im Hauptmenü der Startknopf angeklickt wird, werden alle Raum- und Einrichtungsobjekte geladen. Dies wird mit der \verb+reset_game()+-Funktion gemacht. 

\begin{Verbatim}
def reset_game():
    global uv_lamp, cipher_wheel, scene, rooms_list, room, color_theme
    import random, Tools, Room, Object, Cryptography

# die Zeit im Timer wird zurückgesetzt auf die Startzeit
    timer.time = default_timer_time 
# Die UV-Taschenlampe und die Chiffrierscheibe werden neu geladen
    uv_lamp = Tools.UVLight()
    cipher_wheel = Tools.CipherWheel() 
# Das Spiel wird gestartet und es wird zur Hauptansicht umgeschaltet
    scene = 1 
# Die Zeit auf dem Timer wird gestartet
    timer.running = True 

# alle Codes, Passwörter und Schlüssel für alle Räume werden geladen
    room1_code = str(random.randint(1000, 9999))
    room2_code = random.choice(words_list)
    room2_key = str(random.randint(1, 25))
    room3_code = random.choice(words_list)
    room3_key = random.choice(words_list) 

# Alle Räume mit ihren Objekten werden generiert
    rooms_list = [ 
        Room.Room([
            Object.Shelf(0, 1),
            Object.Door(5, 0, room1_code),
            Object.Shelf(1),
            Object.Lamp(4),
            Object.Desk(2),
            Object.Desk(3, 1, room1_code)
        ]),
        Room.Room([
            Object.Door(0, 1, room2_code),
            Object.Shelf(5),
            Object.Lamp(1),
            Object.Desk(4, 2),
            Object.Calendar(2, room2_code, room2_key),
            Object.Desk(3)
        ]),
        Room.Room([
            Object.Desk(0, 4, room3_code),
            Object.Shelf(5),
            Object.Desk(1, 3),
            Object.Shelf(4, 2, room3_key),
            Object.Door(2, 0, '55555'),
            Object.Lamp(3)
        ])
    ]

# Der Raum wird zum ersten Raum in der Raumliste gesetzt
    room = rooms_list[0] 
# Das Farbschema wird so gesetzt, dass die Lampe im Raum eingeschalten ist
    color_theme = light_theme
\end{Verbatim}

Um einen neuen Raum einzuprogrammieren, muss also im Grunde genommen beim Laden aller Codes, Passwörter und Schlüssel der Code oder das Passwort des neuen Raumes eingefügt werden, der neue Raum mit seinen Objekten in die \verb+rooms_list+ eingefügt werden und eventuelle neue Einrichtungsgegenstände müssen in der \verb+Objects.py+ Datei definiert werden.